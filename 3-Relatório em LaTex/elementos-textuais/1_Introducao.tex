\chapter{Introdução}

Até a década de 70, a região do Oeste da Bahia permaneceu como um imenso território de reserva, parcialmente ocupado e com baixo nível de atividade econômica. A
partir dos anos 80, a região enfrentou uma rápida aceleração no ciclo de desenvolvimento, principalmente no que se refere as atividades que envolvem a agropecuária, como a criação de gado, produção de grãos e a fruticultura. Apenas na atividade agrícola, entre os anos de 1985 e 2020, a região teve uma expansão de mais de 10 vezes da área colhida, passando de aproximadamente 200 mil hectares em 1985 para cerca de 2,2 milhões de hectares em 2020, segundo os dados da pesquisa de Produção Agrícola Municipal (PAM) \cite{ibge2021pam}. Com esse acelerado crescimento das mudanças de uso e cobertura da terra, ter informações espacialmente explícitas sobre essas mudanças é fundamental para o planejamento e gestão sustentável dos recursos naturais, formulação de políticas públicas, gestão dos recursos hídricos, previsão da produção agrícola, entre outras aplicações sociais.

\renewcommand{\cleardoublepage}{}
\renewcommand{\clearpage}{}
\vspace{5mm}