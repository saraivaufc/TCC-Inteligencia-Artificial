\chapter{Contextualização}

Nos últimos anos, com o aumento na disponibilidade de imagens de satélite gratuitas, capacidade de processamento e eficientes técnicas de inteligência artificial, houve um forte crescimento na utilização de métodos automáticos para o mapeamento do uso e cobertura da terra. No Brasil, os dois projetos mais reconhecidos com esta finalidade são o TerraClass, do Instituto Nacional de Pesquisas Espaciais (INPE) \cite{coutinho2013uso} e o MapBiomas \cite{souza2020reconstructing}. O TerraClass produziu mapas anuais do uso e cobertura da terra para os biomas Amazônia e Cerrado em alguns anos, já o MapBiomas produz mapas anuais, para todo o território brasileiro, desde de 1985 até 2020, em sua última versão. Para mapeamentos anuais, os dois projetos cumprem muito bem o seu papel, de gerar informações anuais de qualidade sobre o uso e cobertura da terra. Porém, quando precisamos de informações intra-anuais, ou seja, identificar mudanças que ocorrem de um mês para o outro e até de uma semana para a outra, tanto o TerraClass quanto o MapBiomas não conseguem nos fornecer essas informações. Na agricultura, por exemplo, esses projetos conseguem nos dizer se uma determinada região foi cultivada ou não em um determinado ano, porém não conseguem nós dizer quantos ciclos de plantio/colheita essa região teve em um único ano-safra e nem quais culturas foram plantadas em cada ciclo, informação essencial para muitos tipos de aplicações, como a gestão de recursos hídricos e previsão da produção agrícola. 

\renewcommand{\cleardoublepage}{}
\renewcommand{\clearpage}{}
\vspace{5mm}