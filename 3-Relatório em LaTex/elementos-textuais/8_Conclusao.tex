\chapter{Conclusão}

Neste trabalho, desenvolvemos uma metodologia que permite a geração de mapas do uso e cobertura da terra para a mesorregião do Extremo Oeste Baiano, no estado da Bahia, com uma periodicidade de 16 dias. A abordagem desenvolvida foi capaz de mapear treze classes de uso e cobertura da terra dentro da área de estudo, ao mesmo tempo em que permite acompanhar, a cada dezesseis dias, as mudanças que possam vir a ocorrer entre essas classes, como por exemplo, a mudança da classe 'exposed\_soil' para 'soybean' indicando o momento do plantio da soja e a mudança da classe 'soybean' para 'exposed\_soil' indicando o momento da colheita da soja. Aplicamos diversas técnicas para a compatibilização e correção das imagens de satélite, o que nós possibilitou trabalhar com imagens Landsat 8 e Sentinel 2A e 2B como se fossem uma única coleção de imagens de satélite. Treinamos três diferentes modelos de aprendizado de máquina, dois baseados no algoritmo \textit{Random Forest} e um baseado em redes neurais recorrentes utilizando a arquitetura de redes neurais \textit{Long short-term memory - LSTM}. O modelo baseado em LSTM foi o que apresentou os melhores resultados, com um \textit{F1-Score} médio de 79\% nas quatorze classes mapeadas (treze classes de uso e cobertura mais a classe de plano de fundo). O método apresentado aqui pode ser usado para mapear grandes áreas de uma forma relativamente rápida e a um custo relativamente baixo, comparado à interpretação manual de imagens de satélite feita por especialista. Para isso, o modelo baseado em LSTM, que apresentou os melhores resultados, precisaria ser retreinado com mais amostras coletadas na região de interesse para permitir que o modelo entenda toda a variabilidade das classes na nova região.

\renewcommand{\cleardoublepage}{}
\renewcommand{\clearpage}{}
\vspace{5mm}