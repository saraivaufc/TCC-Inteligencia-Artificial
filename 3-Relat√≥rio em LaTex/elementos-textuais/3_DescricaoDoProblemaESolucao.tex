\chapter{Descrição do Problema e da Solução Proposta}

Diante desse contexto, este trabalho tem como objetivo desenvolver uma metodologia para permitir a geração de mapas do uso e cobertura da terra na mesorregião do Extremo Oeste Baiano, no estado da Bahia, com uma periodicidade de 16 dias. 

Para facilitar o entendimento do problema e da solução proposta, utilizamos a técnica dos 5W's, que consiste em responder as seguintes perguntas:

\textbf{Why?}: Identificar as variações intra-anuais do uso e cobertura da terra, como ciclos de plantio e colheita na agricultura, é fundamental para o planejamento e a gestão sustentável dos recursos naturais, formulação de políticas, gestão dos recursos hídricos, previsão da produção agrícola, entre outras aplicações sociais.

\textbf{Who?}: As imagens foram obtidas do satélite Landsat 8, disponibilizadas pelo Serviço Geológico dos Estados Unidos (USGS, pela sigla em inglês), e dos satélites Sentinel 2A e 2B, gerenciados pela Agência Espacial Europeia (ESA, pela sigla em inglês). Os polígonos anotados com as classes de uso e cobertura que foram utilizados para o treinamento e validação dos modelos foram obtidos do LEM Dataset (\citeonline{sanches2018lem}, \citeonline{oldoni2020lem}). 

\textbf{What?}: Desenvolver uma metodologia que permita a geração de mapas do uso e cobertura da terra com uma periodicidade de 16 dias.

\textbf{Where?}: Mesorregião do Extremo Oeste Bahiano, no estado da Bahia.  

\textbf{When?}: Os dados anotados do LEM Dataset foram obtidos nos períodos de Junho de 2017 à Maio de 2018, e de Outubro de 2019 à Setembro de 2020. Consequentemente, utilizamos esses mesmos períodos para obter as imagens de satélite e realizar o treinamento e validação dos modelos.  


\renewcommand{\cleardoublepage}{}
\renewcommand{\clearpage}{}
\vspace{5mm}